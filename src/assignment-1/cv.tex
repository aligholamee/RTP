%%%%%%%%%%%%%%%%%%%%%%%%%%%%%%%%%%%%%%%%%
% Medium Length Graduate Curriculum Vitae
% LaTeX Template
% Version 1.1 (9/12/12)
%
% This template has been downloaded from:
% http://www.LaTeXTemplates.com
%
% Original author:
% Rensselaer Polytechnic Institute (http://www.rpi.edu/dept/arc/training/latex/resumes/)
%
% Important note:
% This template requires the res.cls file to be in the same directory as the
% .tex file. The res.cls file provides the resume style used for structuring the
% document.
%
%%%%%%%%%%%%%%%%%%%%%%%%%%%%%%%%%%%%%%%%%

%----------------------------------------------------------------------------------------
%	PACKAGES AND OTHER DOCUMENT CONFIGURATIONS
%----------------------------------------------------------------------------------------

\documentclass[margin, 10pt]{res} % Use the res.cls style, the font size can be changed to 11pt or 12pt here


    \usepackage{helvet} % Default font is the helvetica postscript font
    \usepackage[shortlabels]{enumitem}
    %\usepackage{newcent} % To change the default font to the new century schoolbook postscript font uncomment this line and comment the one above
    \usepackage{url}
	\usepackage{color,hyperref}
	\definecolor{darkblue}{rgb}{0.0,0.0,0.35}
	\hypersetup{colorlinks,breaklinks,
		linkcolor=darkblue,urlcolor=darkblue,
		anchorcolor=darkblue,citecolor=darkblue}
    \setlength{\textwidth}{5.1in} % Text width of the document
    \newlist{innerlist}{itemize}{3}
    \setlist[innerlist]{label=\enskip\textbullet,leftmargin=*,parsep=0pt,itemsep=0pt,topsep=0pt,partopsep=0pt}
    \begin{document}
    
    %----------------------------------------------------------------------------------------
    %	NAME AND ADDRESS SECTION
    %----------------------------------------------------------------------------------------
    
    \moveleft.5\hoffset\centerline{\huge\bf Ali Gholami} % Your name at the top
     
    \moveleft\hoffset\vbox{\hrule width\resumewidth height 1.5pt}\smallskip % Horizontal line after name; adjust line thickness by changing the '1pt'
     
    \moveleft.5\hoffset\centerline{\textsc{Computer Engineering \& Information Technology Department}} % Your address
    \moveleft.5\hoffset\centerline{\textsc{Amirkabir University of Technology}}
 	\moveleft.5\hoffset\centerline{\url{aligholamee@aut.ac.ir}}
	\moveleft.5\hoffset\centerline{\url{ceit.aut.ac.ir/~aligholamee}}
	\moveleft.5\hoffset\centerline{\url{github.com/aligholamee}}
	
    %----------------------------------------------------------------------------------------
        
    \begin{resume}
    
    %----------------------------------------------------------------------------------------
    %	EDUCATION SECTION
    %----------------------------------------------------------------------------------------
     
    \section{EDUCATION}  
    \textbf{B.S. Computer Engineering} \textsc{@}
	   \href{http://aut.ac.ir/aut/}{\textsc{Amirkabir University of Technology}}\\
	   {\UrlFont[Global Rank of 97 in CE] @ USNEWS}\hfill {\UrlFont{\underline{GPA: 3.6/4}}}\\
	   {\UrlFont[National Rank of 2] @ ARWU}
 	\vspace{0.1cm}
 	
    \textbf{Mathematics \& Physics Diploma} \textsc{@}
    \href{http://www.kamal.sch.ir/}{\textsc{Kamal Highschool}}\hfill {\UrlFont{\underline{GPA: 19/20}}}
  
  	\vspace{0.5cm}
  	
  	\section{RELATED \\ COURSES}
	\textbf{Machine Learning} \textsc{@} \textsc{Amirkabir University of Technology}\\
	\textbf{Computer Vision} \textsc{@} \textsc{Udacity}\\
	\textbf{Deep Learning} \textsc{@} \textsc{Udacity}\\
	\textbf{cs231n} \textsc{@} \textsc{Stanford University}\\
    %----------------------------------------------------------------------------------------
    %	Technology SKILLS SECTION
    %----------------------------------------------------------------------------------------
    
    \section{RESEARCH \\ EXPERIENCE} 

	\textbf{CEIT} \textsc{@} \textsc{Amirkabir University of Technology}
	\hfill {Dec 2018 -- Present}\\
	\textit{Computer Vision | Pattern Recognition}
	\vspace{0.15cm}
	\begin{innerlist}
		\item Implementation of \textit{AlexNet CNN} architecture using \textit{Tensorflow}.
		
		\item Implementation of a \textit{DCGAN} to draw \textit{MNIST} characters using \textit{Tensorflow}.
		
		\item Implementation of a \textit{Variational Autoencoder} 	using \textit{Tensorflow}.
		
		\item Implementation of various \textit{Deep Learning} techniques using \textit{Tensorflow}.
		
	\end{innerlist}

  	\vspace{0.5cm}
    %----------------------------------------------------------------------------------------
    %	PAPERS AND TECHNICAL REPORTS SECTION
    %----------------------------------------------------------------------------------------
     
    \section{TECHNICAL REPORTS}
    \textbf{Design \& Implementation of Programming Languages}\\
    \textit{Advisor: Prof. Mehran S. Fallah} -- 
    \href{https://github.com/aligholamee/HALFLIFE/tree/master/reports}{\UrlFont[docs]}
    
    \textbf{Machine Learning}\\
    \textit{Advisor: Prof. Mohamad E. Shiri} --
    \href{https://github.com/aligholamee/Hornburg}{\UrlFont[docs]}
    
    \textbf{Microprocessors \& Assembly Programming}\\
    \textit{Advisor: Prof. Mahdi Homayounpour} --
    \href{https://github.com/aligholamee/Microprocessors}{\UrlFont[docs]}

  	\vspace{0.5cm}
    %----------------------------------------------------------------------------------------
    %	WORK EXPERIENCE
    %---------------------------------------------------------------------------------------- 
    
    \section{WORK \\ EXPERIENCE}
	\textbf{Internship} \textsc{@}
	\href{https://www.arvancloud.com/}{\textsc{Arvan Cloud}}
	\hfill {Jun -- Sep 2017}\\
	\textit{Web Application Development}
	\begin{innerlist}
		\item \textit{HTML, CSS, PHP, Laravel, Javascript, ECMAScript, Node.js, Vue.js, React.js}
	\end{innerlist}
	
	\textbf{Internship} \textsc{@}
	\href{http://www.fandogh.org/}{\textsc{Fandogh}}
	\hfill {Jun -- Aug 2017}\\
	\textit{Mobile Application Development}
	\begin{innerlist}
		\item \textit{Java, React Native}
	\end{innerlist}

  	\vspace{0.5cm}
    %----------------------------------------------------------------------------------------
    %	TEACHING EXPERIENCE
    %----------------------------------------------------------------------------------------
    
    \section{TEACHING \\ EXPERIENCE} 
    
   \textbf{T.A.} \textsc{@}
   \textsc{CEIT} \textsc{@} \textsc{Amirkabir University of Technology}
   \hfill {Sep -- Dec 2017}\\
   \textit{Microprocessors \& Assembly Programming}\\
   \textit{Advisor: Prof. Mahdi Homayounpour}
   
   \textbf{T.A.} \textsc{@}
   \textsc{ENG} \textsc{@} \textsc{Kharazmi University of Tehran}
   \hfill {Sep -- Dec 2015}\\
   \textit{Foundations of Programming in C++}\\
   \textit{Advisor: Dr. Azadeh Mansouri}
      	\vspace{0.5cm}
    %----------------------------------------------------------------------------------------
	%	TALKS
	%----------------------------------------------------------------------------------------
	\section{TALKS} 
	\textbf{Machine Learning at Scale}
	\hfill {Oct 2017}
		\vspace{0.15cm}
	\begin{innerlist}
		\item Based on the paper \textit{Rules of Machine Learning} by \href{http://martin.zinkevich.org/rules_of_ml/rules_of_ml.pdf}{Dr. Martin Zinkevich}.
	\end{innerlist}

	\textbf{Energy Awareness}
	\hfill {July 2017}
		\vspace{0.15cm}
	\begin{innerlist}
		\item Based on the paper \textit{Energy-aware adaptation for mobile applications} by \href{http://www-cgi.cs.cmu.edu/afs/cs.cmu.edu/Web/People/odyssey/docdir/s17.pdf5}{Dr. Jason Flinn}.
	\end{innerlist}

	\textbf{Metasploit Framework}
	\hfill {May 2017}
		\vspace{0.15cm}
	\begin{innerlist}
		\item Introduction to \textit{Metasploit Framework} \& \textit{Social Engineering} techniques.
	\end{innerlist}
  	\vspace{0.5cm}
    %----------------------------------------------------------------------------------------
	%	HONORS
	%----------------------------------------------------------------------------------------
	\section{HONORS}
	\textbf{Admitted} to 
	\textbf{Amirkabir University of Technology}
	among all\hfill {Aug 2018}\\ bachelor students at Computer Engineering 
	Department,\\ Kharazmi University of Tehran.
	
	\textbf{Ranked top 3}
	among all bachelor students at Computer Engieering \hfill {July 2016}\\ Department, Kharazmi University of Tehran.
	
	\textbf{Ranked top 0.006}
	in the Nationwide University Entrance Exam \hfill {July 2014}\\ among all students in 
	Mathemathics and physics (approximately 250,000).
    \end{resume}
  	\vspace{0.5cm}
    %----------------------------------------------------------------------------------------
	%	COMPETENCES
	%----------------------------------------------------------------------------------------
	\section{SKILLS}
	
	\textbf{Languages }
	Persian (\emph{native}), English (\emph{advanced working proficiency})
	
	\textbf{Programming}
	\textit{Python, VHDL, C/C++, Java, ARM Assembly, AVR Assembly, Javascript, HTML/CSS, \LaTeX, Racket, ML, Scheme}.
	
	\textbf{Tools \& Platforms}
	\textit{Tensorflow, scikit-learn, Numpy, Pandas, Matplotlib, Weka, Arduino, ARM, AVR, CodeVision, Xillinx Vivado, ModelSim, Atmel Studio, Cadence PSpice, Keil, Dr. Racket, MongoDB, PostgreSQL, MySQL, Visual Studio, TeXstudio}.
	  	\vspace{0.5cm}
    %----------------------------------------------------------------------------------------
	%	PROJECTS
	%----------------------------------------------------------------------------------------
\section{NOTABLE \\ PROJECTS}

\textbf{Annealing}, 
\textit{Data Cleaning \& Preprocessing}
\begin{innerlist}
	\item Preprocessing and cleaning the dataset of annealing. Reached 98\% accuracy. \href{https://github.com/aligholamee/Datadigger/assignment-2/src}{\UrlFont[code]}\href{https://github.com/aligholamee/Datadigger/assignment-2/docs}{\UrlFont[report]}
\end{innerlist}

\textbf{Titanic}, 
\textit{Data Science \& Feature Engineering}
\begin{innerlist}
	\item Prediction of Titanic survivals as a part of Kaggle competition. Reached an Accuracy of 83\% and Recall of 76\%. \href{https://github.com/aligholamee/Titanic}{\UrlFont[notebook]}
\end{innerlist}

\textbf{MNIST-Drawer}, 
\textit{Variational Autoencoder}
\begin{innerlist}
	\item Implementation of a \textit{Variational Autoencoder} to draw \textit{MNIST} dataset characters using \textit{Tensorflow}. \href{https://github.com/aligholamee/MNIST-Drawer}{\UrlFont[code]}
\end{innerlist}

\textbf{notMNIST}, 
{\textit{Convolutional Neural Network}
	\begin{innerlist}
		\item Implementation of multiple machine learning classifiers and regularization techniques on the \textit{notMNIST} dataset using \textit{Tensorflow}. \href{https://github.com/aligholamee/notMNIST}{\UrlFont[code]}
	\end{innerlist}

	\textbf{Freeman}, 
	{\textit{Hardware Programming \& Co-design}
		\begin{innerlist}
			\item Implementation of a \textit{Parking Controller} \& \textit{Security Controller} using \textit{VHDL}. \href{https://github.com/aligholamee/Freeman}{\UrlFont[code]}
		\end{innerlist}
		
		\textbf{Numex}, 
		\textit{Functional Programming}
		\begin{innerlist}
			\item Implementation of an \textit{Advanced Functional Interpreter} using \textit{Racket}. \href{https://github.com/aligholamee/NUMEX}{\UrlFont[code]}
		\end{innerlist}
		
		\textbf{Hornburg}, 
		\textit{Deep Learning Basics}
		\begin{innerlist}
			\item Implementation of \textit{Principal Machine Learning Algorithms} using \textit{Python}. \href{https://github.com/aligholamee/Hornburg}{\UrlFont[code]}
		\end{innerlist}
		
		\textbf{Iris}, 
		\textit{Multi-nomial classification}
		\begin{innerlist}
			\item Multi-nomial classification of \textit{Iris} dataset using \textit{scikit-learn}. \href{https://github.com/aligholamee/IRIS}{\UrlFont[code]}
		\end{innerlist}
	
		\textbf{ARMHE}, 
		\textit{Advanced RISC Machine Programming}
		\begin{innerlist}
			\item Implementation of the \textit{Histogram Equalization} algorithm on the \textit{STMF32F407VGT6} with \textit{ARMv4T} architecture using \textit{ARM Assembly}. \href{https://github.com/aligholamee/ARMHE}{\UrlFont[code]}
		\end{innerlist}

		\textbf{Cinder}, 
		{\textit{Low Level Programming}
			\begin{innerlist}
				\item Implementation of a basic \textit{Operating System} with \textit{C}. \href{https://github.com/aligholamee/CinderOS}{\UrlFont[code]}
			\end{innerlist}

			\textbf{Sockets}, 
			\textit{Socket Programming}
			\begin{innerlist}
				\item Implementation of various types of \textit{Sockets} in \textit{Interprocess Communication} \& \textit{TCP/IP Protocol} with \textit{C}. \href{https://github.com/aligholamee/Socket-Programming-Package}{\UrlFont[code]}
			\end{innerlist}

			\textbf{Toofan}, 
			\textit{Android Application Development}
			\begin{innerlist}
				\item Implementation of a \textit{Weather Forecast Application} on the \textit{Android} platform using \textit{Java} \& \textit{Android Studio}. \href{https://github.com/aligholamee/Toofan}{\UrlFont[code]}
			\end{innerlist}

			\textbf{Huffman}, 
			\textit{Huffman Coding}
			\begin{innerlist}
				\item Implementation of the \textit{Huffman Text Compression Algorithm} using \textit{Java}. \href{https://github.com/aligholamee/Java-Huffman-Zipper}{\UrlFont[code]}
			\end{innerlist}
			
			\textbf{2048}, 
			\textit{C++ Programming}
			\begin{innerlist}
				\item Implementation of the \textit{2048 Puzzle Game} with various gameplay tweaks using \textit{C++}. \href{https://github.com/aligholamee/2048-Console-Game}{\UrlFont[code]}
			\end{innerlist}
			
			\textbf{Manobase}, 
			\textit{VHDL Programming}
			\begin{innerlist}
				\item Implementation of the \textit{Morris Mano's Base Computer} using \textit{VHDL}. \href{https://github.com/aligholamee/Mano-Basic-Computer-Design}{\UrlFont[code]}
			\end{innerlist}
	
    \end{document}