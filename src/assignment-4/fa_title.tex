%% -!TEX root = AUTthesis.tex
% در این فایل، عنوان پایان‌نامه، مشخصات خود، متن تقدیمی‌، ستایش، سپاس‌گزاری و چکیده پایان‌نامه را به فارسی، وارد کنید.
% توجه داشته باشید که جدول حاوی مشخصات پروژه/پایان‌نامه/رساله و همچنین، مشخصات داخل آن، به طور خودکار، درج می‌شود.
%%%%%%%%%%%%%%%%%%%%%%%%%%%%%%%%%%%%
% دانشکده، آموزشکده و یا پژوهشکده  خود را وارد کنید
\faculty{دانشکده مهندسی کامپیوتر و فناوری اطلاعات}
% گرایش و گروه آموزشی خود را وارد کنید
\department{گرایش نرم افزار}
% عنوان پایان‌نامه را وارد کنید
\fatitle{پرسش و پاسخ بصری با استفاده از شبکه های عصبی کانولوشنی عمیق و بازگشتی}
% نام استاد(ان) راهنما را وارد کنید
\firstsupervisor{دکتر محمد رحمتی}
%\secondsupervisor{استاد راهنمای دوم}
% نام استاد(دان) مشاور را وارد کنید. چنانچه استاد مشاور ندارید، دستور پایین را غیرفعال کنید.
% نام نویسنده را وارد کنید
\name{علی }
% نام خانوادگی نویسنده را وارد کنید
\surname{غلامی}
%%%%%%%%%%%%%%%%%%%%%%%%%%%%%%%%%%
\thesisdate{فروردین 1396}

% چکیده پایان‌نامه را وارد کنید
\fa-abstract{
در این پروژه، مدلی مبتنی بر شبکه های عصبی کانولوشنی و بازگشتی عمیق، به منظور پرسش و پاسخ بصری ارائه شده است. درک ماشین از تصاویری که به آن ارائه می شود، هیچ گاه با درک انسان از تصاویر قابل قیاس نبوده است. به همین دلیل، پژوهش های انجام گرفته در بینایی ماشین، همواره در صدد بهبود این قابلیت در ماشین ها بوده است. برای این منظور، ویژگی های مهم تصاویر استخراج شده و سپس در مورد وجود یا عدم وجود اشیا مورد نظر در تصاویر تصمیم گیری انجام می گیرد. فراتر از این موضوع، قابلیت درک ماشین از سوالات انسان درباره این تصاویر و پاسخ به این سوالات بر اساس ویژگی ها و دانش مستخرج از تصاویر  است که به تازگی مورد توجه پژوهشگران عرصه ی بینایی ماشین قرار گرفته است.
}


% کلمات کلیدی پایان‌نامه را وارد کنید
\keywords{پرسش و پاسخ بصری، شبکه های کانولوشنی عمیق، شبکه های بازگشتی عمیق، پردازش زبان طبیعی، پردازش تصویر)}



\AUTtitle
%%%%%%%%%%%%%%%%%%%%%%%%%%%%%%%%%%