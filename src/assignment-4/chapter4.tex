\chapter{پیاده سازی، آزمون و ارزیابی}
در این فصل از گزارش، ابتدا کلیاتی در رابطه با پیاده سازی پروژه را بیان خواهیم نمود. بعلاوه، مجموعه داده مورد استفاده در آموزش و ارزیابی شبکه را بررسی کرده و همچنین مدل پیشنهادی برای ارزیابی دقت پاسخ های تولیدی را بررسی خواهیم کرد.
\section{پیاده سازی }
سیستم پرسش و پاسخ بصری مطرح در این گزارش، به زبان پایتون(نسخه 3.6) و با استفاده از چارچوب کاری تنسورفلو پیاده سازی شده است. هسته این چارچوب کاری به زبان سی پلاس پلاس و با استفاده از پلتفرم توسعه موازی کودا پیاده سازی شده است. این چارچوب کاری توسط تیم گوگل برین در حال توسعه بوده و پشتیبانی می شود.\\
ایده ی اصلی مطرح در این چارچوب کاری، بیان محاسبات در قالب گراف است. هر گره ی این گراف، یک واحد محاسباتی را مشخص می کند. با این رویکرد می توان شبکه های پیچیده را به راحتی پیاده سازی نمود.

\section{معیار بلیو}
این معیار، یکی از معیار های ارزیابی مدل های ترجمه ماشینی است که در حوزه پرسش و پاسخ بصری نیز مورد استفاده قرار می گیرد. در حوزه ترجمه ماشینی، ترجمه های مختلفی از یک جمله در زبان مبدا، می توان در زبان مقصد ارائه داد. برای تشخیص بهترین ترجمه بین ترجمه های کاندید برای یک حمله در زبا ن مبدا، می توان از این معیار استفاده نمود.


\section{معیار سایدر}
این معیار در بین پژوهشگران حوزه تولید خودکار شرح بر تصاویر ارائه شده است. این معیار در سال 2015 توسط ودانتام و همکارانش ارائه شد. هدف اصلی این معیار این است که توافق شرح های مرجع تولید شده توسط انسان را یافته و سپس میزان انطباق شرح تولید شده خودکار با این توافق را اندازه گیری نماید.







