\chapter{جمع‌بندي و نتيجه‌گيري و پیشنهادات}
%%%%%%%%%%%%%%%%%%%%%%%%%%%%%%%%%%%%%%%%%%%
تولید و ذخیره سازي روزافزون تصاویر، سهولت در استفاده از دوربین هاي تصویر برداري و گوشی هاي موبایل،
دسترسی آسان به اینترنت در تمام نقاط شهر و افزایش تعداد شبکه هاي اجتماعی و نرم افزارهاي موبایل، باعث
افزایش نیاز کاربران به سامانه هاي هوشمند مدیریت تصاویر شده است. سامانه هایی که علاوه بر مدیریت ذخیره و
بازیابی تصاویر، قدرت دسته بندي خودکار، جستجوي محتوایی، درك و توصیف تصاویر از هر موضوعی باشند.
ارائه مدل هاي هوشمند که بتوانند به طور خودکار براي هر تصویري، توصیف متناظر در قالب جملات زبان طبیعی
تولید کنند، از جمله مهم ترین اقدامات در راستاي رسیدن به سامانه مدیریت تصاویر به شمار می رود.\\
در سال هاي بعد از  2007، می توان گفت توجه پژوهش گران بیشتر به سمت مدل هاي تولید جمله جلب شد و
چالش هاي موجود دراین حوزه که اغلب بدون راه حل بودند یا با راه حل هاي ابتدایی حل می شدند، بیش از پیش
مورد استقبال پژوهش گران قرار گرفتند. مدل هاي مختلفی براي تولید جمله به کار گرفته شد. از جمله این مدل ها
می توان به روش هاي موجود در حوزه تولید زبان طبیعی، بازیابی شبیه ترین جمله موجود در مجموعه داده و استفاده
از کلیشه زبانی، اشاره کرد. اما هیچ یک از این روش ها، نتوانستند تمام معضلات را حل نمایند.\\
اما با حل مشکل ناپایداري آموزش شبکه هاي عصبی بازگشتی در سال  2011 توسط هینتون، فصل جدیدي در
حوزه تولید جمله در این مساله شروع شد. شبکه هاي عصبی بازگشتی، ابزارهاي قدرتمندي در کاربرد پیش بینی
دنباله هاي زمانی و تولید جمله به شمار می روند. قابلیت هاي بالاي این مدل ها، پژوهش گران را برآن داشت که
تمامی روش هاي گذشته را کنار گذاشته و تماما از شبکه هاي عصبی بازگشتی براي تولید جمله استفاده نمایند.
استفاده از شبکه هاي عصبی بازگشتی، ذهن اغلب پژوهش گران را به سمت استفاده از شبکه هاي کانولوشنی عمیق
در مرحله استخراج اطلاعات از تصاویر می کشاند. شبکه هاي عصبی کانولوشنی عمیق، در استخراج ویژگی هاي
بسیار خوب از تصاویر، قدرت بالایی دارند. از حدود سال  2014 به بعد و با جابجایی مدل هاي گرافی احتمالی
با شبکه هاي عصبی کانولوشنی عمیق، صفر تا صد فرایند تولید خودکار شرح بر تصاویر، با استفاده از شبکه هاي
عصبی و یادگیري عمیق انجام می شد.