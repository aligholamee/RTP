\chapter{مروری بر مطالعات گذشته}
\section{مقدمه}
دراین بخش به بررسی راهکار های مختلف ارائه شده در موضوغ پرسش و پاسخ بصری می پردازیم. پرسش و پاسخ بصری، یکی از چالش های بزرگ روز در عرصه ی هوش مصنوعی می باشد. بسیاری از ایده های مطرح در حل این چالش ها، از نجوه ی عملکرد ذهن انسان مدل برداری شده است.
\subsection{مغز چگونه تصاویر را درک می کند؟}
تصاویر در مغز انسان به طرز خارق العاده ی پردازش می شوند. به گونه ای که در اولین نگاه می توان بیشترین اطلاعات موجود در تصویر را استخراج کرد و آنها را توصیف نمود. بعلاوه، قدرت مغز در تخلیل شنیدار و پاسخ به این تحلیل و نیز صحنه ی دریافتی از طریق بینایی، همواره قابل تامل و مطالعه می باشد.\\
این ایده که مغز قادر است تا حجم زیادی از اطلاعات را به سرعت پردازش کرده و در مورد آنها تصمیم اتخاذ کند، از جانب پژوهشگران مورد بررسی قرار گرفته است. به عنوان مثال \cite{potter1976short} پژوهشی است که در آن تعدادی از تصاویر به صورت دنباله ای به افرادی نشان داده می شود و نیز توصیفاتی از طرف آنها ارائه می گردد. پژوهشگران با انجام این آزمایش پی بردند که مغز قادر است در کمتر از 200 میلی ثانیه به صحنه های دریافتی پاسخ دهد.
در پژوهش \cite{fei2007we} آزمایش دیگری انجام شده است که از اهمیت بسیاری بر خوردار است. در پژوهش های قبلی، افرادی که تصاویر را توصیف می کردند، درباره موضوغ کلی تصاویر اطلاعاتی داشتند.اما در این آزمایش، تصاویر مختلفی از دنیای واقعی که محدود به شرایط خاضی نبوده اند، بدون ارائه پیش فرض درباره ی موضوع، به افراد نمایش داده شده و از آنها خواسته شده که تصویر را به بهترین شکل توصیف کنند. نتایج بدست آمده به صورت زیر می باشد:\\
1. حداکثر زمان لازم برای مغز انسان به منظور درک صحنه، برابر با 500 میلی ثانیه می باشد.\\
2. این مدت زمان، برای صحنه های ساده و بدون پیچیدگی، به حدود 100 میلی ثانیه می رسد.\\
 
 \index{کتاب}
\index{پارسی‌لاتک}
\index{بی‌دی}
\index{سوال}
\index{عنصر}
\index{گزینه}
\index{ژاکت}
\index{مرکز دانلود}
\index{اجرا}
\index{تک‌لایو}
\index{ثالث}
\index{جهان}
\index{چهار}
\index{حمایت}
\index{خواهش}
\index{دنیا}
\index{زی‌پرشین}
\index{ریحان}
\index{شیرین}
\index{صمیمی}
\index{ضمیر}
\index{طبیب}