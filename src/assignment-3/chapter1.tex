\chapter{مقدمه}

\section{مقدمه}
درک انسان از محیط اطراف خود، از ابتدای کودکی شکل می گیرد. اگر از یک کودک، درباره جهانی که برای او قابل رویت است، سوالی پرسیده شود، بلافاصله درباره ی این جهان جملاتی توصیفی با دقت بسیار بالا ارائه می کند. انتقال این قدرت به ماشین ها از این جهات مختلفی حائز اهمیت می باشد. یکی از جهات اصلی آن، کاربرد های بی شماری است که می توان از این قابلیت در امور مختلفی استفاده کرد. دوربین های نظارتی، خودروهای خودران و ... نمونه هایی از این کاربرد ها هستند. در این فصل ابتدا موضوع پژوهش را به طور کامل بیان کرده و اهمیت ارائه راهکار مناسب در این مورد را بررسی می کنیم.
سپس رویکرد های مختلف را برای حل این مساله بیان می کنیم.
\section{توضیح مساله}\label{sec2}
درک تصاویر به همراه درک سوالاتی که مرتبط با آن تصاویر پرسیده می شود، می تواند یکی از اساسی ترین قابلیت های یک سامانه مدیریت هوشمند تصاویر، خودروی خودران و یا یک سیستم بازیابی تصویر هوشمند باشد.
برای این منظور نیاز است که ابتدا صحنه به نمایش درآمده توسط ماشین هضم گردد. بدین معنی که بدون درک درست از آنچه در تصویر موجود است، ساخت چنین سامانه ای امکان پذیر نخواهد بود. 

\section{مروری بر روند پژوهش های پیشین}
روش هایی که جهت حل مسائل پرسش و پاسخ بصری ارائه شده اند از تنوع بالایی برخوردار هستند. اگرچه، می توان اغلب این روش ها را در راستای حل چهار چالش مهم زیر در نظر گرفت:\\
1. چالش استخراج ویژگی از تصاویر\\
2. چالش استخراج ویژگی از متن(سوالات پرسیده شده)\\
3. چالش ترکیب بردار های ویژگی مستخرج از تصویر و متن\\
4. چالش تولید پاسخ متناسب با ویژگی ها ترکیب شده\\
\\
با تحول عظیمی که در سال 2012 با ارائه مدل کانولوشنی عمیق جهت استخراج ویژگی ارائه شد و نیز افزایش قدرت پردازشی ماشین ها، توجه به سمت شبکه های کانولوشنی جهت استخراج ویژگی از تصاویر بیشتر شد. با پیشرفت این مدل ها و ترکیب آنها با مدل های پردازش زبان طبیعی، تولید شرح بر تصاویر نیز مورد توجه قرار گرفت. از سال 2015 تا به اکنون، یکی از جالب ترین موضوعاتی که توجه پژوهشگران را به سمت خود جلب کرده است، موضوع پرسش و پاسخ بصری می باشد. این موضوع ارتباط تنگاتنگی با موضوع تولید شرح بر تصاویر دارد. به همین منظور، بسیاری از تکنییک های رایج در بحث شرح بر تصاویر، در موضوع پرسش و پاسخ بصری نیز مورد توجه قرار گرفته است.