\chapter{مقدمه}

\section{مقدمه}
درک انسان از محیط اطراف خود، از ابتدای کودکی شکل می گیرد. اگر از یک کودک، درباره جهانی که برای او قابل رویت است، سوالی پرسیده شود، بلافاصله درباره ی این جهان جملاتی توصیفی با دقت بسیار بالا ارائه می کند. انتقال این قدرت به ماشین ها از این جهات مختلفی حائز اهمیت می باشد. یکی از جهات اصلی آن، کاربرد های بی شماری است که می توان از این قابلیت در امور مختلفی استفاده کرد. دوربین های نظارتی، خودروهای خودران و ... نمونه هایی از این کاربرد ها هستند. در این فصل ابتدا موضوع پژوهش را به طور کامل بیان کرده و اهمیت ارائه راهکار مناسب در این مورد را بررسی می کنیم.
سپس رویکرد های مختلف را برای حل این مساله بیان می کنیم.
\section{توضیح مساله}\label{sec2}
درک تصاویر به همراه درک سوالاتی که مرتبط با آن تصاویر پرسیده می شود، می تواند یکی از اساسی ترین قابلیت های یک سامانه مدیریت هوشمند تصاویر، خودروی خودران و یا یک سیستم بازیابی تصویر هوشمند باشد.
برای این منظور نیاز است که ابتدا صحنه به نمایش درآمده توسط ماشین هضم گردد. بدین معنی که بدون درک درست از آنچه در تصویر موجود است، ساخت چنین سامانه ای امکان پذیر نخواهد بود. 

\section{مروری بر پژوهش های پیشین}

بزودی مطالب دقیقتری در مورد پژوهش های پیشین پیرامون این موضوع ارائه می گردد.