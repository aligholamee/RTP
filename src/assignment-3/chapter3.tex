\chapter{روش ارائه شده در پژوهش}
%\thispagestyle{empty}

\section{مقدمه}
در این فصل باید روشی جهت پیاده سازی یا نحوه ی انجام تحقیقات صورت گرفته ارائه گردد. این روش باید جزییات ویژه ای را پوشش دهد، چالش های مطرح را حل و فصل و نیز نوآوری هایی ارائه دهد. بنابراین، ابتدا به مشکلات اصلی موجود در این روش و سپس به ارائه راه حل های آن می پردازیم.

\section{بیان مشکلات موجود}
1. تعداد زیاد پارامتر های شبکه عصبی و نیاز به توان پردازشی بالا
تعداد پارامتر های یک شبکه ی عصبی بسیار بالاست. این تعداد در مدل  الکسنت به 15 میلیون پارامتر هم می رسد. جالب آن است که بیشتر از 95 درصد این پارامتر ها مربوط به 3 لایه ی تمام متصل انتهای شبکه می باشد. در صورتی که اندازه و کیفیت تصاویر نیز بالاتر رود، تعداد این پارامتر ها به صورت تصاعدی افزایش می یابد. در بحث استخراج ویژگی از تصاویر جهت دریافت جزییات صحنه، ارائه راهکار جدید برای معماری شبکه عصبی همواره مورد توجه پژوهشگران بوده است.\\
2. یکی دیگر از مشکلات موجود، نحوه ی ارزیابی جملات و پاسخ هایی است که ماشین تولید می کند. این پاسخ ها اغلب با توصیفاتی که انسان در مورد تصاویر ارائه می دهد، از لحاظ جزییات بسیار فاصله دارد. به عنوان مثال، در یکی از معیار های مطرح شده برای ارزیابی پاسخ ها، جملات تولید شده توسط ماشین توسط سه داور انسانی بررسی میشود و هریک نمره ای به آن جمله می دهد. بر این اساس دقت جملات تولید شده ارزیابی می گردد. هرچند، این روش از دقت ارزیابی پایینی برخوردار می باشد. بنابراین یکی از چالش ها همواره، ایجاد یک راهکار دقیقتر برای ارزیابی بوده است.